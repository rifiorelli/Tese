%%%%%%%%%%%%%%%%%%%%%%%%%%%%%%%%%%%%%
%% Master Thesis - Computer Engineering
%% Copyright 2009 Ricardo Alexandre Fiorelli, Erick Poletto
%% This document is distributed by the terms of the license
%% included in the file LICENCE.
%%%%%%%%%%%%%%%%%%%%%%%%%%%%%%%%%%%%%

%%%%%%%%%%%%%%%%%%%%%%%%%%%%%%%%%%%%%
%% Fourth Chapter
%% Analysis and Results
%%%%%%%%%%%%%%%%%%%%%%%%%%%%%%%%%%%%%

\chapter{Analysis and Results} \label{chap4:analysis_results}
%TODO explicar o que é este capitulo, pra que ele serve, realmente PROCURAR na internet o que geralmente se coloca num capitulo assim
%TODO estudar e escrever o que aprendeu
%
    The objective of this chapter is to describe the designed component database and evaluate it against measured data and data obtained from the component manufacturers. This evaluation will determine the adequacy of power consumption data provided by the database as a means to estimate the power consumption of different machine configurations.
\section{Analysis} \label{sec4:analysis}
%TODO here it is explained the database, how it was built, the database schema and etc\ldots
%TODO
\subsection{Component Database} \label{sec4:component_database}
    The component database will be described in terms of its relations. For each relation it's semantic meaning will be explained along with the data it contains and the source of this data. The database schema is present in Figure~\ref{fig:database_schema}.

    \begin{figure}[h!tb]
        \centering
        \includegraphics[scale=0.6]{graphics/database_schema}
        \caption{Component database schema}
        \label{fig:database_schema}
    \end{figure}

    \subsubsection*{Classification relations}
        In order to provide an efficient classification of the components, four category relations were used, namely 1CAT, 2CAT, 3CAT and DEVICE. In this hierarchical disposition each component is classified in three levels by the relations 1CAT, 2CAT and 3CAT and then uniquely identified by the DEVICE relation. This classification follows the one present in the website analysed by the crawler\footnote{http://www.sisoftware.net/}.

        \paragraph*{Relation 1CAT}
            The first level of classification is very general as it just states the final usage of the component. As the components used in the database are computer components they will be defined as such in this first level.

        \paragraph*{Relation 2CAT}
            This relation fits the component into one of the main devices of a computer architecture. These are the following: processors, main memory, mass storage device, networking devices, CD/DVD player, mainboard, etc.

        \paragraph*{Relation 3CAT}
            The component is classified in a last level to provide contextual information about its usage. Processors, for example, are defined either as boxed or directly available in an OEM computer configuration. Memory can be used for servers, workstations or notebooks. Network devices can be LAN cards or networking equipment such as routers. Finally, mass storage devices are separated by technology, which can be hard-disk or flash-based.

        \paragraph*{Relation DEVICE}
            The last level is the identification of each component by an unique MPN number. This is the central relation in the database as the pricing, benchmark and characteristic relations are associated to specific components. The MPN information stored here was obtained with the use of the WEBSPHINX crawler over the SiSoftware shopping site\footnote{http://www.sisoftware.net/}. This device relation uses MPN as its key and associates for each component the 3 previous classifications, the component model/name and the date in which it was inserted in the database.
            
    \subsubsection*{Characteristic relations}
        This set of relations associate with each component its specifications, provided by the manufacturer. As SiSoftware Sandra provides these, it was be preferred source for this kind of data. Other sources such as CPU-Z reports and component datasheets were used in case the information could not be found in Sandra.
        
        \paragraph*{Relation DEVICE_CHARACTERISTIC}
            This determines the value of the characteristic being described. The value in char_value can be for example number of cores or clock frequency for a processor, disk rotation or average bandwidth for a hard drive or size and bandwidth for main memory. The attribute CHAR_ID is linked to the following relation in order to define the type of the characteristic.
            
        \paragraph*{Relation CHARACTERISTIC}
            Each characteristic, other than a value associated in the previous relation, should be described in terms of type and measurement unit. One example may be the clock frequency which is measured in MHz. In this relation the attribute CHAR_NAME would be ``clock frequency'', the attribute CHAR_UM would be ``MHz'' and the DESCRIPTION attribute would explain the meaning of the characteristic.
            
        \paragraph*{Relation SOURCE}
            For each component, the source of their characteristics could be either SiSoftware Sandra, the Web or other softwares such as CPU-Z.
                        
\subsection{Overview} \label{sec4:analysis_overview}
    % explicar que a gente pegou as coisas e separou por tabelas, tentar nao ser redundate com o que teve no cap3

\section{Results} \label{sec4:results}
%TODO falar que aqui a gente vai explicar os resultados de todos os métodos
%TODO 
\subsection{Benchmark Results} \label{sec4:benchmark_results}

