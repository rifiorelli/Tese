%%%%%%%%%%%%%%%%%%%%%%%%%%%%%%%%%%%%%
%% Master Thesis - Computer Engineering
%% Copyright 2009 Ricardo Alexandre Fiorelli, Erick Poletto
%% This document is distributed by the terms of the license
%% included in the file LICENCE.
%%%%%%%%%%%%%%%%%%%%%%%%%%%%%%%%%%%%%

%%%%%%%%%%%%%%%%%%%%%%%%%%%%%%%%%%%%%
%% First Chapter
%% Introduction
%%%%%%%%%%%%%%%%%%%%%%%%%%%%%%%%%%%%%


\chapter{Introduction} \label{chap1:introduction}


%\begin{quotation}{\footnotesize\noindent{\emph{``Every body perseveres in its state of being at rest or of moving uniformly straight ahead, except insofar as it is compelled to change its state by forces impressed.''}}\begin{flushright}Cohen \& Whitman 1999\end{flushright}}\end{quotation}%XXX Nao sei se deixa a citacao ai ou nao...

%     This is a general introduction to what the thesis is all about – it is ust a description of the contents
% of each section. Briefly summarize the question (you will be stating the question in detail later),
% some of the reasons why it is a worthwhile question, and perhaps give an overview of your main
% results. This is a birds-eye view of the answers to the main questions answered in the thesis (see
% above).

\section{Motivation} \label{sec1:motivation}
% Mostrar o que levou a realiza��o do trabalho, e as motiva��es para que o problema seja entendido.
% Situa��o do leitor no contexto
%O greenict tem sido muito importatne para o mundo, as empresas, depois da crise economica, comecaram a ver que é importante, pois, reduz muitos custos, além de contribuirem apra o meio ambiente\ldots bla bla bla

    The planet is threatened by global warming. The progressive pressure we impose to the environment, has already exceeded the limits imposed by the available natural resources. In actual figures, it can be stated that 125\% of the renewal capacity of natural resources is currently consumed. If the growth continues at this rate, in 2050 we will be consuming more than twice the production capacity of the planet Earth~\cite{Townsend:2002:2050}. 
    
    The search for environmental friendly solutions is spreading throughout all the economic sectors and consumers have become more aware of environmental issues and thus opt for the products and services of companies which have proven to be more ecologically friendly. Moreover, it is possible to say that soon we will enter in a green trend - if we are not already in it - where environmental policies will be executed without economical or political pressure, but rather as a necessary measure sustainability of the business. In a following phase companies will perceive green solutions as a competitive differential instead of a necessary preoccupation thereby giving the green issue a push towards being part of the business.
        
    In the last years, the concept - and idea - of Green ICT has been increasing, it is the mantra of "Going Green". A study from Info-Tech Research Group~\cite{info-tech07}, made in U.S.A., believes that the increasing interest to adopt green solution will start to generate more meaningful actions. However, there exists a big gap between what companies think it is a green ICT solution and what they are really doing about it. The study predicts ``Info-Tech expects continued interest in green IT strategies and significant traction among those initiatives that both reduce waste and reduce cost. As enterprises begin to translate concern for green into practice, we expect higher spending in many leading areas, such as data center design, virtualization and consolidation, print optimization, and system management tools''.
    
    As a new trend, there exists a high interest, yet low adoption of green initiatives. Nonetheless, companies have reached the consensus that it is extremely necessary to start changing their minds towards green thoughts. So, leading industries and governments have started a proactive going green idea and offering green solutions in order to create a market around it. This include the allocation of a significant amount of money in researches in the area. Besides that, it is important to notice that as soon as the information in the direction of well management of the energy consumed start to spread, companies will start to be more worried about it. And the first step will be to start to count environmental harm and power consumption in the TCO accountancies.
    
    Furthermore, there are direct benefits when adopting a Green ICT solutions. The most important benefit is the economic aspect. Some studies says that the costs with power and cooling in data centers can reach 20\% of the IT cost. In the economic sector, the potential savings for companies could be huge. And simple actions can have big impacts in the organization, for example, "A bank branch could save nearly \euro40,000 (\$53,020) just by turning off phones and wireless access points outside business hours," David Frampton, VP general manager of Cisco's LAN switching business unit, told Reuters~\cite{Chestney2009}. Last year, Symantec launched a study named ``State of the Data Center Report 2008'', which the social responsibility was the least important reason for applying a Green initiative. It states that ``reduction of costs'' and ``reduction of power consumption'' are the most important reasons when investing in such idea. 
    
    Another benefit is that Green ICT can be used as a marketing strategy, with the growing popularity, vendor have started to put green labels on their products and consumers also started to seek for green products in contrast to traditional ones. It has helped to give some companies competitive advantages. And if the scenario continues like that, the companies which do not adopt the ``Going Green'' concept will lose this competitive advantage, while spending more economic resources.
    
    The most difficult phase when applying a new solution is going against the inertia of the company. In this case, the state of being at rest is not applying a green solution and it is needed the force and influence for pushing forward this idea, as soon as the first step is taken the others come with time. A critical issue for taking the first step apropos the green initiative is the allocation of budget. Before the economical crisis IT resources could be larger, but now it has been more difficult to manage the budget towards new initiatives. Therefore, the Green ICT budget should reflect what the company is expecting from the solution. Even with the economical crisis, there is an increase with expenses with technology production, mainly when allied with productivity increase and reduction of cost related wastes. 
    
    In conclusion, one of the most important interests when ``Going Green'' is focused on ``Green Data Center'', which includes re-engineering of the Data Center location, the resources used, with virtualization, database consolidation and efficient use of energy in servers. 

%    Portanto, as empresas que investem em datacenter especializados podem esperar um contínuo processo de modernização, já que este mercado investe constantemente em soluções de consolidação e virtualização de servidores, storage e equipamentos de rede, além de blade, a grande tendência do momento.

% \section{Terminology Clarification} \label{sec1:terminology_clarification}
%TODO  OPTIONAL
%     A brief section giving background information may be necessary, especially if your work spans
% two or more traditional fields. That means that your readers may not have any experience with
% some of the material needed to follow your thesis, so you need to give it to them. A different title
% than that given above is usually better; e.g., ”Frammis Algebra.”


\section{Definition of the problem} \label{sec1:problem}
    % eu usei catalog na tese inteira, inglês EUA, catalogue é ingles britanico.
    The present study was conducted in order to empirically and quantitatively catalog computer components related data by means of benchmarking, web research and to validate these with the use of direct measurement. This information can then be used to better compose the data center with respect to energy efficiency. The goal is to support companies in order to apply a green solution by assisting the choice of the right combination of components.
    
    Moreover, this study may provide a means to identify critical bottlenecks in power consumption and to address the problem by making a more efficient use of the identified components. 
    
    \subsection{Thesis Statement}\label{sec1:thesis_statement}
        The study aims to address the following research questions:
        \begin{enumerate}
	        \item Which computer component spends less energy?
	        \item What is the best configuration, concerning power efficiency?
	        \item How to catalog, analyze and understand the power efficiency in a component.
        \end{enumerate}

\section{Solution Strategy} \label{sec1:solution_strategy}
%TODO
    The first step to be taken in the direction of the solution to the problem is data collection. In this phase, it was acquired from many sources, that will be described in Chapter~\ref{chap3:methodology}, data related to the energy consumed by the components. After this, the information collected was separated by categories and linked together through a MPN code, which is unique for each component. All the information acquired from the measures was inserted into a consolidated database of components. The next step is the analysis of these data collected. It is aimed to find \emph{qualitative solution} using the \emph{database of components} in order to exploit the best configuration and component concerning its energy efficiency.

\section{Structure} \label{sec1:structure}
%TODO
% colocar a estrutura dos capitulos e do documento sussa...
    This document is structured as follows:
    \begin{itemize}
        \item Chapter~\ref{chap1:introduction} is the introduction;
        \item Chapter~\ref{chap2:state_of_the_art} is the state of the art, giving relevant information presenting the major ideas of the work;
        \item Chapter~\ref{chap3:methodology} is the Methodology, where the problem is engineered, it is explained the research design used and the justification and discussion of the method;
        \item Chapter~\ref{chap4:analysis_results} is the Analysis and Results part, it is stated how the database was composed and a critical analysis of the results;
        \item Chapter~\ref{conclusion} is the conclusion. It presents the conclusion of the work that has been developed and suggests how this work could be further developed.
    \end{itemize}
    
    
