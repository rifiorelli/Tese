%%%%%%%%%%%%%%%%%%%%%%%%%%%%%%%%%%%%%
%% Master Thesis - Computer Engineering
%% Copyright 2009 Ricardo Alexandre Fiorelli, Erick Poletto
%% This document is distributed by the terms of the license
%% included in the file LICENCE.
%%%%%%%%%%%%%%%%%%%%%%%%%%%%%%%%%%%%%

%%%%%%%%%%%%%%%%%%%%%%%%%%%%%%%%%%%%%
%% First Chapter
%% Introduction
%%%%%%%%%%%%%%%%%%%%%%%%%%%%%%%%%%%%%


\chapter{Introduction} \label{chap1:introduction}
Purpose of the study.
%     This is a general introduction to what the thesis is all about – it is ust a description of the contents
% of each section. Briefly summarize the question (you will be stating the question in detail later),
% some of the reasons why it is a worthwhile question, and perhaps give an overview of your main
% results. This is a birds-eye view of the answers to the main questions answered in the thesis (see
% above).

\section{Motivation} \label{sec1:motivation}
%TODO
% Mostrar o que levou a realiza��o do trabalho, e as motiva��es para que o problema seja entendido.
% Situa��o do leitor no contexto
%O greenict tem sido muito importatne para o mundo, as empresas, depois da crise economica, comecaram a ver que é importante, pois, reduz muitos custos, além de contribuirem apra o meio ambiente\ldots bla bla bla
    The concept of Green ICT have been really important 

% \section{Terminology Clarification} \label{sec1:terminology_clarification}
%TODO  OPTIONAL
%     A brief section giving background information may be necessary, especially if your work spans
% two or more traditional fields. That means that your readers may not have any experience with
% some of the material needed to follow your thesis, so you need to give it to them. A different title
% than that given above is usually better; e.g., ”Frammis Algebra.”


\section{Definition of the problem} \label{sec1:problem}
%TODO

\section{Solution Strategy} \label{sec1:solution_strategy}
%TODO

\section{Structure} \label{sec1:structure}
%TODO
% colocar a estrutura dos capitulos e do documento sussa...
    This document is structured as follows:
    \begin{itemize}
    	\item Chapter~\ref{chap1:introduction} is the introduction;
    	\item Chapter~\ref{chap2:state_of_the_art} is the state of the art, giving relevant information presenting the major ideas of the work;
    	\item Chapter~\ref{chap3:methodology} is the Methodology, where the problem is engineered, the justification and discussion of the method;
    	\item Chapter~\ref{chap4:analysis_results} is the Analysis and Results part, it is stated how the problem is solved and 
    	\item 
    \end{itemize}
