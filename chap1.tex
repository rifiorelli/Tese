%%%%%%%%%%%%%%%%%%%%%%%%%%%%%%%%%%%%%
%% Master Thesis - Computer Engineering
%% Copyright 2009 Ricardo Alexandre Fiorelli, Erick Poletto
%% This document is distributed by the terms of the license
%% included in the file LICENCE.
%%%%%%%%%%%%%%%%%%%%%%%%%%%%%%%%%%%%%

%%%%%%%%%%%%%%%%%%%%%%%%%%%%%%%%%%%%%
%% First Chapter
%% Introduction
%%%%%%%%%%%%%%%%%%%%%%%%%%%%%%%%%%%%%


\chapter{Introduction} \label{chap1:introduction}


%     This is a general introduction to what the thesis is all about – it is ust a description of the contents
% of each section. Briefly summarize the question (you will be stating the question in detail later),
% some of the reasons why it is a worthwhile question, and perhaps give an overview of your main
% results. This is a birds-eye view of the answers to the main questions answered in the thesis (see
% above).

\section{Motivation} \label{sec1:motivation}
% Mostrar o que levou a realiza��o do trabalho, e as motiva��es para que o problema seja entendido.
% Situa��o do leitor no contexto
%O greenict tem sido muito importatne para o mundo, as empresas, depois da crise economica, comecaram a ver que é importante, pois, reduz muitos custos, além de contribuirem apra o meio ambiente\ldots bla bla bla

    The planet is threatened by global warming. The progressive pressure we impose to the environment, has already exceeded the limits imposed by the available natural resources. In actual figures, it can be stated that, each year, 125\% of the renewal capacity of natural resources is currently consumed. If the growth continues at this rate, in 2050 we will be consuming more than twice the production capacity of the planet Earth~\cite{Townsend:2002:2050}. 
    
    The seek for environmental friendly solutions is spreading throughout all the economic sectors and consumers have become more aware of environmental issues and thus opt for products and services of companies which have proven to be more ecologically friendly. Moreover, it is possible to say that soon we will enter in a green trend - if we are not in already - where environmental policies will be executed without economical or political pressure, but rather as a necessary measure sustainability of the business. In a following phase companies will perceive green solutions as a competitive differential instead of a necessary preoccupation thereby giving the green issue a push towards being part of the business.
        
    In the last years, the concept of Green ICT has been increasingly popular by the mantra of \emph{Going Green}. A study from Info-Tech Research Group~\cite{info-tech07}, made in U.S.A., believes that the increasing interest in adopting a green solution is beginning to generate meaningful actions. However, there exists a big gap between what companies think it is a green ICT solution and what they are really doing about it. This same study states that ``Info-Tech expects continued interest in green IT strategies and significant traction among those initiatives that both reduce waste and reduce cost. As enterprises begin to translate concern for green into practice, we expect higher spending in many leading areas, such as data center design, virtualization and consolidation, print optimization, and system management tools''.
    
    As a new trend, there exists a high interest, yet low adoption. Nonetheless, companies have reached the consensus that it is extremely necessary to start changing their minds towards green thoughts. So, leading industries and governments have started a proactive \emph{Going Green} promotion to expand the existent market. Measures taken in favor of this green market include the allocation of a significant amount of money in researches in the area. Besides that, it is important to notice that as soon as information related of careful management of energy consumption starts to spread, it will start to attract the companies attention. The first step will be to start to study the impact of environmental harm and power consumption in the TCO accountancies.
    
    The most important benefit of a Green ICT strategy is the reduction of costs related to the energy consumption. Some studies says that the costs with power and cooling in data centers can reach up to 20\% of the IT cost. In the economic sector, the potential savings for companies could be huge and simple actions can have big impacts in the organization. As David Frampton, VP general manager of Cisco's LAN switching business unit, explained to Reuters~\cite{Chestney2009}, "a bank branch could save nearly \euro40,000 (\$53,020) just by turning off phones and wireless access points outside business hours". Following the same pattern, last year, Symantec launched a study named ``State of the Data Center Report 2008'', in which the social responsibility was the least important reason for applying a Green initiative. It states that \emph{reduction of costs} and \emph{reduction of power consumption} are the most important reasons to invest in such idea. 
    
    Another benefit is that Green ICT can be used as a \emph{marketing strategy} and with the growing popularity of the issue, vendors have started to put green labels on their products and consumers also started to seek for green products instead of the traditional ones. And if the scenario continues like that, the companies which do not adopt the \emph{Going Green} concept will lose this competitive advantage, while spending more and more economic resources.
    
    The most difficult phase when applying a new solution is going against the inertia of the company. In this case, the state of being at rest is not applying a green solution and what is required is the force and influence for pushing forward the green idea. After the first step has been taken the others should come with time. A critical issue for taking this first step towards the green initiative is the allocation of budget. Before the economical crisis IT resources could be larger, but now it has been more difficult to manage the budget towards new initiatives. Therefore, the Green ICT budget should reflect what the company is expecting from the solution. Even with the economical crisis, there has been an increase in the expenses with technology production related to power efficient products, which usually is allied with productivity increase and reduction of waste generation. 
    
        This work relates to one specific aspect of the \emph{Going Green} concept, which is the \emph{Green Data Center}. The related measures focuses in re-engineering of the Data Center with the use of a wide number of techniques that will be described in the following chapter.

%    Portanto, as empresas que investem em datacenter especializados podem esperar um contínuo processo de modernização, já que este mercado investe constantemente em soluções de consolidação e virtualização de servidores, storage e equipamentos de rede, além de blade, a grande tendência do momento.

% \section{Terminology Clarification} \label{sec1:terminology_clarification}
%TODO  OPTIONAL
%     A brief section giving background information may be necessary, especially if your work spans
% two or more traditional fields. That means that your readers may not have any experience with
% some of the material needed to follow your thesis, so you need to give it to them. A different title
% than that given above is usually better; e.g., ”Frammis Algebra.”


\section{Definition of the problem} \label{sec1:problem}

    The present study was conducted in order to empirically and quantitatively catalog computer components related data by means of benchmarking, web research and to validate these with the use of direct measurement. This information can then be used to better compose the data center with respect to energy efficiency. The goal is to support companies in order to apply a green solution by assisting the choice of the right combination of components and, besides that, point the ones which consumes less energy and have higher productivity (best cost/benefit).
    
    Moreover, this study may provide a means to identify critical bottlenecks in power consumption and to address the problem by making a more efficient use of the identified components. To that end, the aim of this research is to design a computer component database with information about their characteristics and benchmark tests. The information that regards power consumption will then be validated with the use of direct measurements in order to determine the accuracy of the component power estimated by the benchmarking tools.
    
    \subsection{Thesis Statement}\label{sec1:thesis_statement}
        The study aims to address the following questions:
        \begin{enumerate}
	        \item Which computer components are more efficient, i.e. consume the less energy while providing a good performance?
	        \item How to choose among a set of machine configurations the best on, concerning power efficiency?
	        \item How to catalog, analyze and understand the reasons behind power efficiency in a component.
        \end{enumerate}
        
    These will be addressed with the use of an adequate component database, which will be the scope of this work.

\section{Solution Strategy} \label{sec1:solution_strategy}

    The first step to be taken in the direction of the solution to the problem is data collection. In this phase, it was acquired from many sources, that will be described in Chapter~\ref{chap3:methodology}, data related to the energy consumed by the components. After this, the information collected was separated by categories and linked together with a MPN code, which is unique for each component. All the information acquired from the measures was inserted into a consolidated database of components. The next step is the analysis of the collected data. This \emph{database of components} has as objective finding a \emph{qualitative solution} to the issue of choosing the best machine configuration concerning their energy efficiency.

\section{Structure} \label{sec1:structure}

% colocar a estrutura dos capitulos e do documento...
    This document is structured as follows:
    \begin{itemize}
        \item Chapter~\ref{chap1:introduction} is the introduction;
        \item Chapter~\ref{chap2:state_of_the_art} is the state of the art, giving relevant information about available technologies and techniques applied in Green ICT;
        \item Chapter~\ref{chap3:methodology} is the Methodology, where the problem is engineered, the used method used is exposed and a means to evaluate it is provided;
        \item Chapter~\ref{chap4:analysis_results} exposes the results achieved. It describes how the database was created and explains the results of the analysis carried out to evaluate it;
        \item Chapter~\ref{conclusion} is the conclusion. It presents the conclusion of the work and suggests how it could be further developed.
    \end{itemize}
    
    
