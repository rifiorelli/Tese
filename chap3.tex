%%%%%%%%%%%%%%%%%%%%%%%%%%%%%%%%%%%%%
%% Master Thesis - Computer Engineering
%% Copyright 2009 Ricardo Alexandre Fiorelli, Erick Poletto
%% This document is distributed by the terms of the license
%% included in the file LICENCE.
%%%%%%%%%%%%%%%%%%%%%%%%%%%%%%%%%%%%%

%%%%%%%%%%%%%%%%%%%%%%%%%%%%%%%%%%%%%
%% Third Chapter
%% Methodology
%%%%%%%%%%%%%%%%%%%%%%%%%%%%%%%%%%%%%

\chapter{Methodology} \label{chap3:methodology}
    The methodology

%%%%%%%%%%%%%%%%%%%%%%%%%%%%%%%%
% Que porra � essa de Targa??? %
%%%%%%%%%%%%%%%%%%%%%%%%%%%%%%%%
\section{Dati Targa????} \label{sec3:}

\section{SiSoftware SANDRA} \label{sec3:sandra}
    SANDRA was the main software utilized to benchmark the data in this thesis work. It was chosen to be, composing the ones 
    
\subsection{What is} \label{subsec3:whatis_sandra}
\textbf{SiSoftware Sandra} (the \textbf{S}ystem \textbf{AN}alyser, \textbf{D}iagnostic and \textbf{R}eporting \textbf{A}ssistant) is an information \& diagnostic utility. It provides most of the information (including undocumented) one need to know about their hardware, software and other devices whether hardware or software.
    
The software goes beyond the point of other Windows Utilities, by giving the user, the possibility of benchmarking and comparing at both high and low level the computer devices, for example, the CPU, chipset, video adapter, ports, printers, sound card, memory, network, AGP, PCI, PCI-X, PCIe (PCI Express), database, USB, 1394/Firewire, etc.
    
SANDRA makes use of these different modules, which users can take advantage of:

\begin{itemize}
	\item 
\end{itemize}

\section{Other Tools} \label{sec3:other_tools}

\subsection{Tool1} \label{}
\subsection{Tool2} \label{}
\subsection{Tool3} \label{}

\section{Measument Methodology} \label{sec3:measurement_methodology}

