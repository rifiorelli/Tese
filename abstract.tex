%%%%%%%%%%%%%%%%%%%%%%%%%%%%%%%%%%%%%
%% Master Thesis - Computer Engineering
%% Copyright 2009 Ricardo Alexandre Fiorelli, Erick Poletto
%% This document is distributed by the terms of the license
%% included in the file LICENCE.
%%%%%%%%%%%%%%%%%%%%%%%%%%%%%%%%%%%%%

%%%%%%%%%%%%%%%%%%%%%%%%%%%%%%%%%%%%%
%% Abstract
%%%%%%%%%%%%%%%%%%%%%%%%%%%%%%%%%%%%%


\begin{abstract} 
\label{abstract}
% Apresenta��o concisa dos pontos relevantes, dando uma visao rapida e
% clara do conte�do do trabalho.


%%%%%%%%%%%%
%% Importante: Os comentarios abaixo, somente separam cada parte do abstract, porem, nao devem ser 
%% separados, pois trata-se de somente um paragrafo
%%%%%%%%%%%%

%%%Parts of the abstract
%%%Rationale
Since the last few years cost of energy has become an important part of Data Centers total cost of ownership. This has lead organizations worldwide to recognize the importance of a green strategy as a means to save energy.
%%%Objectives
This study was conducted in order to catalog computer components related data by means of benchmarking and web research. To that end a component database was designed and then the power-related data obtained from the used benchmarking tools was validated with the use of direct measurements.
%%%Methods
The data acquired for composing this database of components was obtained from different sources, which were both a benchmarking tool and the web. The benchmarking tool provided information about component performance and power consumption. After that, a web crawler was used to assign each component an unique key and to provide additional information about them.
%%%Results
%%%Conclusions
This work provided a database of components which can be used to support a green ICT methodology by allowing the identification of the most adequate and power-efficient components. It provides for the individuation of energy consumption critical spots in a data center and allows comparisons between several components, which may both be used in data center design.

\end{abstract}


