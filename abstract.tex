%%%%%%%%%%%%%%%%%%%%%%%%%%%%%%%%%%%%%
%% Master Thesis - Computer Engineering
%% Copyright 2009 Ricardo Alexandre Fiorelli, Erick Poletto
%% This document is distributed by the terms of the license
%% included in the file LICENCE.
%%%%%%%%%%%%%%%%%%%%%%%%%%%%%%%%%%%%%

%%%%%%%%%%%%%%%%%%%%%%%%%%%%%%%%%%%%%
%% Abstract
%%%%%%%%%%%%%%%%%%%%%%%%%%%%%%%%%%%%%


\begin{abstract} 
\label{abstract}
% Apresenta��o concisa dos pontos relevantes, dando uma visao rapida e
% clara do conte�do do trabalho.
% TODO

%%%%%%%%%%%%
%% Importante: Os comentarios abaixo, somente separam cada parte do abstract, porem, nao devem ser 
%% separados, pois trata-se de somente um paragrafo
%%%%%%%%%%%%

%%%Parts of the abstract
%%%Rationale
Since few years ago, companies have started to take into the Total Cost of Ownership of the Data Centers the amount of energy consumed. This has been developing an awareness of the importance of how much can be saved when applying a green strategy.
%%%Objectives
This study was conducted in order to empirically and quantitatively catalog computer components related data by means of benchmarking, web research and to validate these with the use of direct measurement. After that, everything should be joint in a database of components for later comparison.
%%%Methods
The data acquired for composing this database of components was obtained from different sources, the sources used were a benchmarking tool, in which data related to the components in different situations were cataloged. After that, a web crawler and a measurement device were used for linking together throughout a unique key and validate the measures obtained with the benchmarking tool, respectively.
%%%Results
%%%Conclusions
Therefore, the use of the database of components can be a powerful tool in the sense that it provides a support to the project with useful information for the comparison and choice of the most adequate and power-efficient for the ICT.

\end{abstract}


