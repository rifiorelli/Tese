%%%%%%%%%%%%%%%%%%%%%%%%%%%%%%%%%%%%%
%% Master Thesis - Computer Engineering
%% Copyright 2009 Ricardo Alexandre Fiorelli, Erick Poletto
%% This document is distributed by the terms of the license
%% included in the file LICENCE.
%%%%%%%%%%%%%%%%%%%%%%%%%%%%%%%%%%%%%

%%%%%%%%%%%%%%%%%%%%%%%%%%%%%%%%%%%%%
%% Conclusions
%%%%%%%%%%%%%%%%%%%%%%%%%%%%%%%%%%%%%

\chapter{Conclusions} \label{conclusion}

%% riassunto problema/strategia soluzione
%% risultati significativi


%%%%%%%%%%%%%%%%%%%%%%%%%%%%%%%%%%%
%%      You generally cover three things in the Conclusions section, and each of these usually merits a
%   separate subsection:
% 6.1 Conclusions
%       Conclusions are not a rambling summary of the thesis: they are short, concise statements of the
%   inferences that you have made because of your work. It helps to organize these as short numbered
%   paragraphs, ordered from most to least important. All conclusions should be directly related to
%   the research question stated in Section 4.
% 6.2 Summary of Contributions
%       The Summary of Contributions will be much sought and carefully read by the examiners. Here
%   you list the contributions of new knowledge that your thesis makes. Of course, the thesis itself
%   must substantiate any claims made here. There is often some overlap with the Conclusions, but
%   that’s okay. Concise numbered paragraphs are again best. Organize from most to least important.
% 6.3 Future Research
%       The Future Research subsection is included so that researchers picking up this work in future
%   have the benefit of the ideas that you generated while you were working on the project. Again,
%   concise numbered paragraphs are usually best.
%%%%%%%%%%%%%%%%%%%%%%%%%%%%%%%%%%%


    This chapter summarizes the main findings of this study and draws out their support for applying a green solution. It thereby aims to enrich the understanding of the method and of the valuable information that can be extracted from the created database.
    
    The use of \emph{green ict} applied to data centers can be a very useful strategy in different scenarios. The database of components resulted from this thesis work can be very effective for what it is proposed to be: offering a way to compare the energy consumption of the computer components in one single place. Retrieving information about how much components spend in terms of power consumption will help the development of the green project in the \emph{Assessment} phase by comparing components already existent in the market with the ones present in the data center. That is important firstly because the analysis and research of power consumption of critical spots can be made with ease. The database can also provide assessment when renewing or expanding the data center by permitting the choice of the most adequate and power-efficient machine configurations.
    
    %XXX Adicionar a conclusao dos resultados da analise dos components
    
    %Another great functionality of the result is that it relates the best case scenario for the ratio performance/power and it adds the price as the third variable for a better component choice.
    In the test conducted with a series of notebooks, the results of the power consumption data analysis proved that the power consumption estimated by the Sandra benchmarks is inaccurate. Although its estimates were incompatible with respect to the direct measurements, the usefulness of the component database should be no lesser. It provides a great number of other useful information, mainly regarding component performance benchmarks and price which are essential when analyzing datacenter equipment.
    
    In this way, for the component database to provide support for a green ICT methodology a new source of power-related data should be found. This could be a systematic measurement of components with an adequate aggregation level or a function of the nominal power found in component specifications.
    
    \pagebreak
    \section{Perspectives and Future Developments}

        As part of the initial phase of the development of a green methodology, there are some possible functionalities and information that could still be added. These are as follows:
    \begin{itemize}
	    \item This first version of the component's database is not automatically updated, so it could be made a back-end robot which would crawl for new components as they are released in the market and automatically populate the database;
	    \item As stated before, the use of a new source of power-related information was suggested. This information should then be inserted in the component database as to make efficiency (performance/power) comparison between components.
	    \item To create a front-end software for a interactive and effective way of comparing the components.
    \end{itemize}







