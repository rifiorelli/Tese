%%%%%%%%%%%%%%%%%%%%%%%%%%%%%%%%%%%%%
%% Master Thesis - Computer Engineering
%% Copyright 2009 Ricardo Alexandre Fiorelli, Erick Poletto
%% This document is distributed by the terms of the license
%% included in the file LICENCE.
%%%%%%%%%%%%%%%%%%%%%%%%%%%%%%%%%%%%%

%%%%%%%%%%%%%%%%%%%%%%%%%%%%%%%%%%%%%
%% Conclusions
%%%%%%%%%%%%%%%%%%%%%%%%%%%%%%%%%%%%%

\chapter{Conclusions} \label{conclusion}

%% riassunto problema/strategia soluzione
%% risultati significativi


%%%%%%%%%%%%%%%%%%%%%%%%%%%%%%%%%%%
%%      You generally cover three things in the Conclusions section, and each of these usually merits a
%   separate subsection:
% 1 Conclusions
%       Conclusions are not a rambling summary of the thesis: they are short, concise statements of the
%   inferences that you have made because of your work. It helps to organize these as short numbered
%   paragraphs, ordered from most to least important. All conclusions should be directly related to
%   the research question stated in Section 4.
% 2 Summary of Contributions
%       The Summary of Contributions will be much sought and carefully read by the examiners. Here
%   you list the contributions of new knowledge that your thesis makes. Of course, the thesis itself
%   must substantiate any claims made here. There is often some overlap with the Conclusions, but
%   that’s okay. Concise numbered paragraphs are again best. Organize from most to least important.
% 3 Future Research
%       The Future Research subsection is included so that researchers picking up this work in future
%   have the benefit of the ideas that you generated while you were working on the project. Again,
%   concise numbered paragraphs are usually best.
%%%%%%%%%%%%%%%%%%%%%%%%%%%%%%%%%%%
    This chapter summarizes the main findings of this study and draws out their support for applying a green solution. It thereby aims to enrich the understanding of the method and the valuable information that can be extracted from this database. 
    
    The use of \emph{green ict} applied to data centers can be a very useful strategy in different scenarios. Using the database of components resulted from this thesis work it can be very effective for what it is proposed to be: offering a way to compare the energy consumption of the computer components in one single place.
    

\section*{Perspectives and Future Developments}
%TODO
Suggestions for future developments, there are

\begin{itemize}
	\item % Link this research with SaaS
	\item %TODO
	\item %TODO
	\item %TODO
	\item %TODO
\end{itemize}
