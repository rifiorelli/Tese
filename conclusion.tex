%%%%%%%%%%%%%%%%%%%%%%%%%%%%%%%%%%%%%
%% Master Thesis - Computer Engineering
%% Copyright 2009 Ricardo Alexandre Fiorelli, Erick Poletto
%% This document is distributed by the terms of the license
%% included in the file LICENCE.
%%%%%%%%%%%%%%%%%%%%%%%%%%%%%%%%%%%%%

%%%%%%%%%%%%%%%%%%%%%%%%%%%%%%%%%%%%%
%% Conclusions
%%%%%%%%%%%%%%%%%%%%%%%%%%%%%%%%%%%%%

\chapter{Conclusions} \label{conclusion}

%% riassunto problema/strategia soluzione
%% risultati significativi


%%%%%%%%%%%%%%%%%%%%%%%%%%%%%%%%%%%
%%      You generally cover three things in the Conclusions section, and each of these usually merits a
%   separate subsection:
% 6.1 Conclusions
%       Conclusions are not a rambling summary of the thesis: they are short, concise statements of the
%   inferences that you have made because of your work. It helps to organize these as short numbered
%   paragraphs, ordered from most to least important. All conclusions should be directly related to
%   the research question stated in Section 4.
% 6.2 Summary of Contributions
%       The Summary of Contributions will be much sought and carefully read by the examiners. Here
%   you list the contributions of new knowledge that your thesis makes. Of course, the thesis itself
%   must substantiate any claims made here. There is often some overlap with the Conclusions, but
%   that’s okay. Concise numbered paragraphs are again best. Organize from most to least important.
% 6.3 Future Research
%       The Future Research subsection is included so that researchers picking up this work in future
%   have the benefit of the ideas that you generated while you were working on the project. Again,
%   concise numbered paragraphs are usually best.
%%%%%%%%%%%%%%%%%%%%%%%%%%%%%%%%%%%


    This chapter summarizes the main findings of this study and draws out their support for applying a green solution. It thereby aims to enrich the understanding of the method and the valuable information that can be extracted from this database. 
    
    The use of \emph{green ict} applied to data centers can be a very useful strategy in different scenarios. Using the database of components resulted from this thesis work it can be very effective for what it is proposed to be: offering a way to compare the energy consumption of the computer components in one single place. Retrieving information about how much components spend, in power consumption, will help the development of the green project in the \emph{Assessment} phase, comparing already existent components in the market with the ones presented in the data center. That is important because the analysis and research of the right and least power consumer component can be made with ease.
    
    %XXX Adicionar a conclusao dos resultados da analise dos components
    
    Another great functionality of the result is that it relates the best case scenario for the ratio performance/power and it adds the price as the third variable for a better component choice.
    
    \pagebreak
    \section{Perspectives and Future Developments}

        As being part of the initial phase of the research and development of a green project, there are many possible solutions that can be adopted when thinking about future developments. The steps are the following:
    \begin{itemize}
	    \item This first version of the component's database is not automatically updated, so, it could be made a back-end robot, which would crawl for new components as they are released in the market and automatically populate the database;
	    \item For the measures with the Energy Measurement Instrument (Section~\ref{sec3:energy_measurement_instrument}), it was performed in 20 different notebooks. It would be interesting to have more measures and with desktops and workstations, in order to make the standard deviation with a better ratio;
	    \item Along with the same idea in the measurement are. With the objective of creating a more substantial result, it would be a good idea to extend the measures for other computer components and with different scenarios;
	    \item To create a front-end software for a interactive and effective way of comparing the components.
    \end{itemize}







