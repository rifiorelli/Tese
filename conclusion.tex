%%%%%%%%%%%%%%%%%%%%%%%%%%%%%%%%%%%%%
%% Master Thesis - Computer Engineering
%% Copyright 2009 Ricardo Alexandre Fiorelli, Erick Poletto
%% This document is distributed by the terms of the license
%% included in the file LICENCE.
%%%%%%%%%%%%%%%%%%%%%%%%%%%%%%%%%%%%%

%%%%%%%%%%%%%%%%%%%%%%%%%%%%%%%%%%%%%
%% Conclusions
%%%%%%%%%%%%%%%%%%%%%%%%%%%%%%%%%%%%%

\chapter{Conclusions} \label{conclusion}

%% riassunto problema/strategia soluzione
%% risultati significativi


%%%%%%%%%%%%%%%%%%%%%%%%%%%%%%%%%%%
%%      You generally cover three things in the Conclusions section, and each of these usually merits a
%   separate subsection:
% 6.1 Conclusions
%       Conclusions are not a rambling summary of the thesis: they are short, concise statements of the
%   inferences that you have made because of your work. It helps to organize these as short numbered
%   paragraphs, ordered from most to least important. All conclusions should be directly related to
%   the research question stated in Section 4.
% 6.2 Summary of Contributions
%       The Summary of Contributions will be much sought and carefully read by the examiners. Here
%   you list the contributions of new knowledge that your thesis makes. Of course, the thesis itself
%   must substantiate any claims made here. There is often some overlap with the Conclusions, but
%   that’s okay. Concise numbered paragraphs are again best. Organize from most to least important.
% 6.3 Future Research
%       The Future Research subsection is included so that researchers picking up this work in future
%   have the benefit of the ideas that you generated while you were working on the project. Again,
%   concise numbered paragraphs are usually best.
%%%%%%%%%%%%%%%%%%%%%%%%%%%%%%%%%%%


    This chapter summarizes the main findings of this study and draws out their support for applying a green solution. It thereby aims to enrich the understanding of the method and of the valuable information that can be extracted from the created database.
    
    The use of \emph{green ict} applied to data centers can be a very useful strategy in different scenarios. The database of components resulted from this thesis work can be very effective for what it is proposed to be: offering a way to compare the energy consumption of the computer components in one single place. Retrieving information about how much components spend in terms of power consumption will help the development of the green project in the \emph{Assessment} phase by comparing components already existent in the market with the ones present in the data center. That is important firstly because the analysis and research of power consumption of critical spots can be made with ease. The database can also provide assessment when renewing or expanding the data center by permitting the choice of the most adequate and power-efficient machine configurations.
    
    In the test conducted with a series of notebooks, the results of the power consumption data analysis shows that the power consumption estimated by the Sandra benchmarks is little different by the power consumed during the measures by hardware. These incompatible results presented in Table~\ref{tab:results_spec_processor} are a result of this difference. They are a reflect of the prior aproximation of the power of the PSU and not taking into account the power of the some components that spend little power, such as memory, wireless card with no network search et al as well as in no sense they are a bad result. So, in order to retrieve a more significant value, the measures should be taken from more computers. Therefore, it is possible to say that the negative measures on the table do not reflect that the measures are inaccurate. On the contrary, they reflect the fact that the measures with the hardware device made with the hardware device takes into account all of the components of the computer and the benchmarking software takes into account only the specific component. And, by stating that, it is possible to say they are inside the standard deviation of the measure, for this amount of computers analyzed.
    
    Although the software estimates were diverse with respect to the direct measurements, the usefulness of the component database should be no lesser. It provides a great number of other useful information, mainly regarding component performance benchmarks and price which are essential when analyzing datacenter equipment. 
    
    In this way, the database of components is a great tool that can provide support to the \emph{Assessment} phase of the green ict initiative with ease.
    
    \pagebreak
    \section{Perspectives and Future Developments}

        As part of the initial phase of the development of a green methodology, there are some possible functionalities and information that could still be added. These are as follows:
    \begin{itemize}
	    \item This first version of the component's database is not automatically updated, so it could be made a back-end robot which would crawl for new components as they are released in the market and automatically populate the database;
	    \item As stated before, the measures should be performed into a higher number of computers. This information should then be inserted in the component database as to make efficiency (performance/power) comparison between components;
	    \item In order to draw a more relevant comparison to the measures taken, it should be made with a more expanded set of situations than the provided by this thesis;
	    \item To create a front-end software for a interactive and effective way of comparing the components.
    \end{itemize}







